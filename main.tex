\documentclass[titlepage]{article}
\usepackage[utf8]{inputenc}
\usepackage{url}

\title{Plato and Art's Education}
\author{Taylor Baum}
\date{August 2018}

\begin{document}
\maketitle

\begin{abstract}
    
\end{abstract}

\section{Introduction}

Arts and humanities programs have long been under attack in the United States education system. "In recent years, financially starved school districts districts have felt compelled to trim or even eliminate arts education. The Great Recession may have accelerated this trend, but according to a 2012 report from the Department of Education, the economic downturn followed a prolonged decline in funding for public school dance and theater programs (the proportion of elementary schools offering these subjects dropped from 20 percent to 3 percent and 4 percent, respectively)" \cite{Mahnken2017}. A massive proponent of this attack comes from the government or education policy makers, who lack . This has become blatantly obvious - if it was not before - with the current administration. When the 2019 federal budget was first proposed, among the organizations that would receive significant budget cuts [were] the National Endowment for the Arts (NEA) and the National Endowment for the Humanities (NEH), both of which supply grant money to arts institutions around America" \cite{Greenberger2018}. While these threats from the Trump administration to dismantle the NEA and NEH did not come to fruition, those who value and have benefited from an education which incorporated the arts should realize that these efforts must be actively fought against.

BeforeAs these programs seem to face increasing pressure and resistance, it is important to ask: is this removal of arts and humanities in lower education ethical? To answer this question, I look to Plato. Plato's commentary on the merits of play and the arts for early education provide a stable logical defense against the removal of these programs.






D'Agnour's paper focuses Plato's position that play could improve the capability of children to learn. If play was incorporated with learning, then the process would no longer feel compulsory, and intellectual development would improve \cite{DAngour2013}.

Manilow writes a paper from his perspective as a student in the Francis W. Parker School, a school which promotes progressive educational techniques inspired by the ideas of Socrates and Plato \cite{Manilow2009}. This will provide interesting insight to the problems which might arise when implementing an Socratic educational system.

In the paper written by Thaler, musical education is explored through a Platonic interpretation. Her main argument states that musical education, in Plato's mind, allows guardians to have a deeper understanding of culture and nature. \cite{Thaler2015}

This paper contains a few works from Plato which focus on the question: of what value is arts education? Within this paper is Book X of the Republic, and Symposium, sections "The Cause and Effect of Love" and "The Ascent of the Soul." I am curious to gain more insight about how the Symposium connects to this idea, as my current understanding does not give me a clear path. \cite{Plato2002}


\bibliography{references}
\bibliographystyle{unsrt}

\end{document}